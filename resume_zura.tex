% NOTE: This is very inspired by http://www.toofishes.net/blog/latex-resume-follow-up/

\documentclass[10pt,letterpaper]{article}
\usepackage[margin=0.75in]{geometry}
\usepackage[utf8]{inputenc}
\usepackage[T1]{fontenc}
\usepackage[stretch=10]{microtype}
\usepackage{hyperref}
\usepackage{tgpagella}
\usepackage{enumitem}
\usepackage{indentfirst}
\pagestyle{empty}

\begin{document}
 
% <header>
\begin{center}
{\huge \textbf{ზურა დოღაძე}}


\vspace{1em}

\href{mailto:z.dogadze2000@gmail.com}{z.dogadze2000@gmail.com} 
\ \ \textbullet \ \ 
\href{https://www.linkedin.com/in/zdogadze/}{linkedin.com/in/zdogadze}
\ \ \textbullet \ \ 
+995 598 44 11 13\end{center}

% </header>

% <work-experience>
\hrule
\vspace{-1.0em}
\subsection*{სამუშაო გამოცდილება}
  \begin{itemize}
    \item[]

    {\textbf{თბილისის სახელმწიფო უნივერსიტეტი} \hfill
    \textbf{თბილისი, საქართველო}}
  \\
  {\emph{საბიუჯეტო კომისიის წევრი - ნახევარ განაკვეთზე} \hfill \emph{ნოემბერი 2022 - სექტემბერი 2023}}

  \begin{itemize}[label=\textbullet]
  \itemsep0.5em
  \item თანამშრომლობა კომიტეტის წევრებთან ფინანსური გამჭვირვალობისა და ანგარიშვალდებულების უზრუნველსაყოფად.
  \item წვლილის შეტანა უნივერსიტეტის რესურსების ეფექტურ მართვაში მონაცემების საფუძველზე.
  \item ხარჯების დაზოგვის შესაძლებლობების იდენტიფიცირება და რესურსების ეფექტურად განაწილება დეპარტამენტებში.
  \item თვეში რამდენიმე შეხვედრა ბიუჯეტის დეტალების შესაფასებლად და ინფორმირებული გადაწყვეტილებების მისაღებად.

  \end{itemize}


   {\textbf{თბილისის სახელმწიფო უნივერსიტეტი} \hfill
      \textbf{თბილისი, საქართველო}}
    \\
    {\emph{სასწავლო მართვის დეპარტამენტის სპეციალისტი - სტაჟიორი სრულ განაკვეთზე} \hfill \emph{მაისი 2022 - აგვისტო 2022}}

    \begin{itemize}[label=\textbullet]
    \itemsep0.5em
    \item სტუდენტთა დახმარება ძირითადი ადმინისტრაციული დოკუმენტების შევსებაში, როგორიცაა მაგისტრატურისა და ბაკალავრიატის დოკუმენტების მოწესრიგებასა/შევსება.
    \item თანამშრომლობა ფაკულტეტის წევრებთან ყოვლისმომცველი სემესტრული განრიგის შედგენაში, მათ შორის გაკვეთილების გეგმების, აკადემიური რესურსების ეფექტური განაწილებისა და კურსების დროული მიწოდების უზრუნველსაყოფად.
    \item ლოჯისტიკური და ტექნიკური მხარდაჭერა გამართული აკადემიური საქმიანობისათვის, როგორიცაა კომპიუტერული სისტემების კონფიგურაცია პროექტორებით, აუდიოვიზუალური აღჭურვილობით და სხვა საჭირო ტექნოლოგიით.
    \item საგანმანათლებლო ღონისძიებებისა და ვორქშოფების ორგანიზება და კოორდინაცია.
    \item ადმინისტრაციული პროცესების დეტალურ ჩანაწერების წარმოება, რაც უზრუნველყოფდა დოკუმენტაციის მთლიანობას და ხელმისაწვდომობას სამომავლო მითითებისთვის.
    \item დეპარტამენტის ყოველდღიურ შეხვედრებში მონაწილეობა სასწავლო პროცესის გაუმჯობესებისა და საგანმანათლებლო სერვისების გაძლიერებასთან დაკავშირებით.
    \end{itemize}

    {\textbf{ICSU - სტუდენტური პარლამენტი} \hfill
      \textbf{თბილისი, საქართველო}}
    \\
    {\emph{თავდაცვისა და უშიშროების კომიტეტის თავჯდომარე - ნახევარ განაკვეთზე} \hfill \emph{სექტემბერი 2018 - მაისი 2021}}

    \begin{itemize}[label=\textbullet]
    \itemsep0.5em

    \item მოქმედ პარლამენტარებთან შეხვედრების ორგანიზება. მუდმივი კომუნიკაციის შენარჩუნება მათ თანაშემწეებთან პირობების მოლაპარაკებისა და თანამშრომლობის ხელშეწყობის მიზნით.
    \item ახალ თაობაში, სამოქალაქო ცნობიერების ამაღლების საინფორმაციო სესიების ხელმძღვანელობა, რომლებიც ასწავლიდნენ სტუდენტებს მათი უფლებებისა და სამოქალაქო პასუხისმგებლობების შესახებ.
    \item 40 ადამიანისგან შემდგარ კომიტეტს მართავა, მათთვის ზედამხედველობის გაწევა, პროდუქტიული გუნდური მუშაობის უზრუნველყოფა.
    \item უშუალო ხელმძღვანელობა საბჭოს სამი წევრისგან შემდგარ გუნდსთვის.
    \item სხვადასხვა სოციალური საკომუნიკაციო ჩატებისა და ჯგუფების მართვა, რაც უზრუნველყოფდა სიახლეების, განახლებების და შესაბამისი ინფორმაციის საზოგადოებისთვის მიწოდებას.
    \end{itemize}

    
\end{itemize}
% </work-experience>

% <education>
\hrule
\vspace{-1.0em}
\subsection*{Education}
  \begin{itemize}
    \parskip=1em
      \vspace{0.05em}

    \item[]
    {\textbf{თბილისის სახელმწიფო უნივერსიტეტი}}
     \hfill
     \textbf{თბილისი, საქართველო}
    \\
    {\emph{ბაკალავრი ისტორიაში}
     \hfill
     \emph{2018 - 2022}}

  \end{itemize}
% </education> 

% <skills>
\hrule
\vspace{-1.0em}
\subsection*{უნარები}
\begin{itemize}
  \parskip=-0.5em
  \vspace{0.05em}
  \item[] \textbf{ენები:} Georgia, English, Russian.
\end{itemize}
% </skills>

\end{document}